\documentclass[ngerman,12pt, titlepage, smallheadings, nomath]{scrartcl}
\usepackage[]{libertine}
\usepackage[german=guillemets]{csquotes}
\usepackage{amssymb,amsmath}
\usepackage{ifxetex,ifluatex}
\usepackage{fixltx2e} % provides \textsubscript
\ifnum 0\ifxetex 1\fi\ifluatex 1\fi=0 % if pdftex
  \usepackage[T1]{fontenc}
  \usepackage[utf8]{inputenc}
\else % if luatex or xelatex
  \usepackage{fontspec}
  \defaultfontfeatures{Ligatures=TeX,Scale=MatchLowercase}
\fi
% use upquote if available, for straight quotes in verbatim environments
\IfFileExists{upquote.sty}{\usepackage{upquote}}{}
% use microtype if available
\IfFileExists{microtype.sty}{%
\usepackage{microtype}
\UseMicrotypeSet[protrusion]{basicmath} % disable protrusion for tt fonts
}{}
\usepackage{hyperref}
\hypersetup{unicode=true,
            pdftitle={Die materielle Grundlage von Rechten und gesellschaftlicher Veränderung},
            pdfauthor={Philipp Schweizer},
            pdfborder={0 0 0},
            breaklinks=true}
\urlstyle{same}  % don't use monospace font for urls
\ifnum 0\ifxetex 1\fi\ifluatex 1\fi=0 % if pdftex
  \usepackage[shorthands=off,main=ngerman]{babel}
\else
  \usepackage{polyglossia}
  \setmainlanguage[]{german}
\fi
\setlength{\emergencystretch}{3em}  % prevent overfull lines
\providecommand{\tightlist}{%
  \setlength{\itemsep}{0pt}\setlength{\parskip}{0pt}}
\setcounter{secnumdepth}{5}
% Redefines (sub)paragraphs to behave more like sections
\ifx\paragraph\undefined\else
\let\oldparagraph\paragraph
\renewcommand{\paragraph}[1]{\oldparagraph{#1}\mbox{}}
\fi
\ifx\subparagraph\undefined\else
\let\oldsubparagraph\subparagraph
\renewcommand{\subparagraph}[1]{\oldsubparagraph{#1}\mbox{}}
\fi
\usepackage[width=13.5cm, height=23cm]{geometry}
\usepackage[doublespacing]{setspace}
\usepackage[JT-Typography]{titlepage}
\setlength\parindent{1.5em}
\renewcommand*{\dictumwidth}{0.5\textwidth}

\title{Die materielle Grundlage von Rechten und gesellschaftlicher Veränderung}
\providecommand{\subtitle}[1]{}
\subtitle{Eine Kritik an David Helds Vorstellung des Kosmopolitanismus}
\author{Philipp Schweizer}
\publishers{Essay im Seminar »Kosmopolitanismus und Modelle globaler Ordnung« von
Dr.~Rosa Sierra}
\place{Goethe-Universität Frankfurt / Main \newline Institut für Philosophie
\newline \vspace{-0.5em} \newline}
\date{SoSe 2016}

\begin{document}
\maketitle

{
\setcounter{tocdepth}{3}
\tableofcontents
}
\newpage

\section{Einleitung}\label{einleitung}

Ausgehend von Marx' Kritik der Menschenrechte, die er in seiner Schrift
\emph{Zur Judenfrage} formuliert (1844/1981a), soll in diesem Essay die
Vorstellung einer kosmopolitanen Ordnung, wie sie bei David Held zum
Ausdruck kommt, kritisch untersucht werden. Dabei konzentriert sich die
Kritik auf seine \emph{Prinzipien einer kosmopolitanen Ordnung} (2013,
S. 65--85). Ziel ist es zu zeigen, dass David Helds Vorstellung des
Kosmopolitanismus keine Lösung der von ihm angeführten globalen Probleme
darstellt. Seine Position ist in entscheidenden Punkten von einer
idealistischen Philosophie geprägt, die es ihm nicht erlaubt, bis zu den
Ursachen dieser Probleme vorzudringen.

Allen Rechten liegt ein Bild des Menschen und seiner Beziehung zur
Gesellschaft zugrunde, das in ihrer Formulierung explizit und implizit
zutage tritt. Dieses Bild kritisiert Marx im Falle der Menschenrechte,
weil es das Individuum nur in Abgrenzung, also \emph{negativ}, zu
anderen Individuen bestimmt. \emph{Gesellschaft} erscheint so als die
Zusammenfassung nicht-gesellschaftlicher, d.i. \emph{asozialer},
Menschen. Der erste Teil dieser Arbeit zeigt, dass Helds
\emph{Prinzipien} in dieser Hinsicht einige Parallelen zu dem Bild
aufweist, welches Marx für die Menschenrechte herausarbeitet. Im zweiten
Teil wird der Frage nachgegangen, \emph{warum}, wie es oben heißt, allen
Rechten ein Bild des Menschen und seiner Beziehung zur Gesellschaft
zugrunde liegt und wie diese Aussage angemessen verstanden werden muss.
Diese Frage verweist auf die philosophische Perspektive in der Marx
Gesellschaft analysiert; -- auf den historischen Materialismus. Der
historische Materialismus führt Rechte und die darin zum Ausdruck
kommenden Ideen auf die gesellschaftlichen Verhältnisse zurück, denen
sie entspringen. Insofern Held das nicht tut, bleibt seine Konzeption
einem voluntaristischem Denken und einem Idealismus verhaftet.

\section{Menschen- und Gesellschaftsbild in negativen
Freiheitsrechten}\label{menschen--und-gesellschaftsbild-in-negativen-freiheitsrechten}

Mit der Schrift \emph{Zur Judenfrage} beginnt Marx eine gründliche
Polemik gegen die Junghegelianer. Er leistet einen wichtigen Beitrag zur
Entlarvung des Antisemitismus, der von den Junghegelianern theoretisch
gestützt wird (vgl. Ullrich, 1974, S. 985f.). Diese behaupten mit Bruno
Bauer an der Spitze, die Juden hätten nicht die gleichen Fähigkeiten wie
die Christen sich politisch zu emanzipieren und müssten daher als
Vorbedingung ihrer Emanzipation ihre Religion aufgeben. Die Juden, so
Bauer, können die allgemeinen Menschenrechte nicht empfangen, weil sie
zu ihrer Religion im Gegensatz stehen. Das für den Juden
charakteristische beschränkte Wesen sondert ihn von den Nichtjuden ab.
Dem widerspricht Marx. Ein Blick auf \enquote{die sogenannten
Menschenrechte, und zwar die Menschenrechte unter ihrer authentischen
Gestalt} zeigt: Die Menschenrechte haben gerade das Recht dieser
Absonderung zu ihrem Kern (1844/1981a, S. 362).

Insbesondere das Menschenrecht der Freiheit ist gerade das Recht sich
abzusondern, \enquote{das Recht des beschränkten, auf sich beschränkten
Individuums.} So heißt es in der Konstitution von 1793 der ersten
französischen Republik (\emph{Acte constitutionnel du 24 juin
1793})\footnote{Marx zieht stellenweise noch die \emph{Deklaration der
  Menschenrechte von 1791} und die Konstitution von 1795 heran. In der
  Frage der Religionsfreiheit zitiert er außerdem aus der
  \emph{Constitution de Pensylvanie} und der \emph{Constitution de
  New-Hampshire}.}:

\begin{quote}
\enquote{Article 6. \enquote{La liberté est le pouvoir qui appartient à
l'homme de faire tout ce qui ne nuit pas aux droits d'autrui}
{[}\ldots{}{]}.} (Zit. n. Marx, 1844/1981a, S. 364)
\end{quote}

\noindent Marx übersetzt, dass die so formulierte Freiheit das Recht
ist, \enquote{alles zu tun und zu treiben, was keinem andern schadet.}
Diese Freiheit ist durch gesetzliche Abgrenzung der Individuen
voneinander bestimmt, sie ist \emph{negative} Freiheit. Sie findet ihre
\enquote{praktische Nutzanwendung} im Menschenrecht des Privateigentums,
formuliert in Artikel 16 eben jener Konstitution von 1793. Es gesteht
dem Menschen zu, \enquote{willkürlich (à son gré), ohne Beziehung auf
andre Menschen, unabhängig von der Gesellschaft, sein Vermögen zu
genießen und über dasselbe zu disponieren}. In dieser
Freiheitsauffassung wird der Mensch \enquote{als isolierte auf sich
zurückgezogene Monade} zur Grundlage genommen, der \enquote{im andern
Menschen nicht die Verwirklichung, sondern vielmehr die Schranke seiner
Freiheit} findet (1844/1981a, S. 364f.). \enquote{Gesellschaft},
schreibt Marx, erscheint so \enquote{als ein den Individuen äußerlicher
Rahmen, als Beschränkung ihrer ursprünglichen Selbstständigkeit. Das
einzige Band, das sie zusammenhält, ist die Naturnotwendigkeit, das
Bedürfnis und das Privatinteresse, die Konservation ihres Eigentums und
ihrer egoistischen Person} (1844/1981a, S. 366).

Was Marx hier kritisiert ist eine liberalistische Auffassung der
Menschenrechte, die mit dem Fokus auf Rechtsgleichheit für tatsächlich
bestehende Ungleichheiten blind sind, sofern diese nicht unmittelbar aus
einer Verletzung der Rechtsgleichheit entspringen (vgl. Lohmann, 1999,
S. 99).\footnote{Lohmann schreibt eine einseitige liberalistische
  Auffassung der Menschenrechte F. A. Hayek und R. Nozick zu, die eine
  solche Position mit Rückgriff auf John Locke vertreten.} Eine solche
Auffassung von Freiheitsrechten und -pflichten ist Helds Sache
allerdings nicht und wir können uns an dieser Stelle nicht damit
begnügen, seine Position als liberalistisch zu kritisieren. In seinen
\emph{Prinzipien} sind negative Rechte und Pflichten um positive
ergänzt.

Betrachten wir zunächst den negativen Gehalt von Helds Prinzipien. Einen
negativen Freiheitsbegriff finden wir in Helds zweitem und drittem
Prinzip, das der \emph{aktiven Handlungsfähigkeit} und das der
\emph{persönlichen Verantwortung und Rechenschaftspflichtigkeit}. Die
aktive Handlungsfähigkeit sieht vor, dass jeder \enquote{Handelnde} die
Pflicht hat, \enquote{sicherzustellen, dass die eigene selbstständige
Handlung \emph{nicht} die Lebenschancen und -möglichkeiten anderer
\emph{beschneidet oder beschränkt}}. Das dritte Prinzip bekräftigt diese
Pflicht, indem es vorsieht, die einzelnen Handelnden für die
Konsequenzen ihres Handelns zur Rechenschaft ziehen zu können (vgl.
2013, S. 67f., Hervorh. von mir). Wir sehen also, dass Held die
\enquote{selbstständige Handlung} negativ bestimmt. Sie hat sich in
einem durch andere selbstständige Handlungen begrenzten Raum zu bewegen.

Die aktive Handlungsfähigkeit wird von Held aber auch in ihrer
Positivität klar zum Ausdruck gebracht, indem sie als die Fähigkeit
gedacht wird, \enquote{die menschliche Gemeinschaft nicht nur zu
akzeptieren, sondern sie auch, im Zusammenhang mit den Entscheidungen
anderer, durch das eigene Handeln zu formen} (2013, S. 67). Das
verdeutlicht Held auch in seiner Zusammenfassung der ersten drei
Prinzipien, deren Kernaussage er unter anderem so bestimmt,
\enquote{dass jede Person fähig ist, im Rahmen der ihr zur Verfügung
stehenden Möglichkeiten autonom zu handeln und dass bei der Entscheidung
darüber, wie gehandelt werden soll oder welche Institutionen geschaffen
werden sollen, die Ansprüche aller betroffenen Personen gleichermaßen
berücksichtigt werden sollten} (2013, S. 71).

Held grenzt sich auch von einer liberalistischen Auslegung des
Eigentumsrechts ab, nach der jeder willkürlich über sein Eigentum
verfügen darf. So wird mit Prinzip drei zwar, \enquote{dass Menschen
sich unweigerlich verschiedene kulturelle, soziale und
\emph{wirtschaftliche} Zielen suchen werden und dass diese Unterschiede
anerkannt werden müssen.} Diese unterschiedlichen Ziele finden ihre
Grenze aber in \enquote{inakzeptablen strukturellen Ungleichheiten}, die
\enquote{manche Menschen daran hindern {[}\ldots{}{]}, ihre vitalsten
Bedürfnisse zu befriedigen} (2013, S. 68).

Wir haben gesehen, dass Marx in seiner Polemik gegen Bauer den
beschränkten Charakter der Freiheitsrechte betont wie er in der
liberalistischen, bürgerlichen Auffassung derselben zum Ausdruck kommt.
Auch bei Held finden wir die Vorstellung des individuellen, auf sich
beschränkten Menschen, der von Held mit einer negativen Formulierung
seiner Rechte und Pflichten Rechnung getragen wird. Auch für Held ist
die Gesellschaft etwas dem Individuum Äußerliches. Diese Sicht auf das
Individuum mündet bei Held allerdings nicht in liberalistische
Konzeption von Freiheit, die gravierenden Ungleichheiten gegenüber blind
wäre. Er zeigt ein Bewusstsein der Probleme dieser Freiheitsauffassung
und begegnet diesen damit, indem er seine Prinzipien um eine positive
Dimension ergänzt.

Allerdings ist es gerade diese positive Dimension, also die Rede von
einer Fähigkeit, autonom zu handeln, die Anlass zu einer tiefer gehenden
Kritik an Held gibt. Denn seine Vorstellung des autonom handelnden
Menschen ist verbunden mit der kantianischen Vorstellung einer
moralischen Autonomie. Gesellschaftliche Verhältnisse sind vom autonomen
Handeln autonomer Individuen im Verbund mit anderen autonomen Individuen
geprägt. Marx lehnt diese Vorstellung nicht nur deshalb ab, weil sie
grundsätzlich das Individuum in den Gegensatz zur Gesellschaft setzt,
sondern auch, weil sie den Menschen nicht in seiner Abhängigkeit von
geschichtlichen und materiellen Verhältnissen fassen kann (vgl. Lohmann,
1999, S. 91f.).

Der zweite Teil soll zeigen, dass diese Vorstellung einer fehlenden oder
falschen Analyse der Dialektik zwischen der Produktionsweise einer
Gesellschaft und der Rechte die sie formuliert zugrunde liegt. Helds
Konzeption, verkommt deshalb zu einem voluntaristischen
Lippenbekenntnis.

\section{Eine historisch-materialistische Kritik an Helds
Kosmopolitanismus}\label{eine-historisch-materialistische-kritik-an-helds-kosmopolitanismus}

Bisher haben wir versucht, Held sein negatives Verständnis der
Freiheitsrechte vorzuhalten und mussten feststellen, dass die Sache so
einfach nicht ist. Um zeigen zu können, dass Helds Prinzipien nicht den
Kern der Probleme treffen, muss über diesen durchaus kritikwürdigen
Punkt hinausgegangen werden. Dazu werden wir zunächst fragen, was sind
\emph{Rechte} und wieso lässt sich in ihnen, wie eingangs behauptet,
immer eine Vorstellung vom Mensch und seinem Verhältnis zur Gesellschaft
finden? Wenn wir diese Frage \emph{materialistisch} beantworten, und das
soll auf den nächsten Seiten erfolgen, wird im Gegensatz dazu deutlich,
dass Held sie \emph{idealistisch} beantwortet und sein Kosmopolitanismus
voluntaristisch, irreführend und als handlungsleitende Idee für eine
fortschrittliche Umwälzung der kapitalistischen Verhältnisse unbrauchbar
ist.

\subsection{Die materielle Grundlage von
Rechten}\label{die-materielle-grundlage-von-rechten}

Im ersten Teil wurde vorausgesetzt, dass sich in Rechten immer ein
Menschenbild und ein Gesellschaftsbild finden lassen. Aber was genau
heißt das und welche philosophische Perspektive liegt dieser Aussage
zugrunde? Diese Prämisse des ersten Teils in zwei philosophische
Richtungen hin interpretierbar. Der philosophische Idealismus geht davon
aus, das die Ursache für Menschen- und Gesellschaftsbild in Rechten in
den Ideen derjenigen zu finden sind, die für diese Rechte direkt oder
indirekt verantwortlich zeichnen. Ideen machen sich durch das Recht
geltend, Ideen stehen am Anfang jeder welterzeugenden Handlung.
Demgegenüber steht der philosophische Materialismus, der diese kausale
Beziehung umkehrt und sie in ihr dialektisches Verhältnis setzt. Das
Sein bestimmt das Bewusstsein. In diesem Satz ist bereits die
Wechselseitigkeit, die Dialektik des Verhältnisse von Sein und
Bewusstsein enthalten. Denn das Bewusstsein ist nichts anderes als
bewusstes \emph{Sein} weshalb auch gilt, dass das Bewusstsein das Sein
bestimmt.

Was sind Rechte, fragt Ash (1964, Kapitel 2) in seiner
historisch-materialistischen Untersuchung moralischer Konzepte. Die
Bedeutung des Begriffs \enquote{Recht}, so seine These, kann durch
Deduktion von bestimmten Rechten gewonnen werden, die in den materiellen
Bedingungen des Lebens dieses oder jenen Typs von Gesellschaft gegründet
sind. Rechte verkörpern Ansprüche, welche die Form von Ansprüchen
gegenüber anderen Menschen oder von Ansprüchen auf Dinge haben können.
Zu sagen, es ist rechtens etwas zu tun, heißt die Legitimität eines
Anspruchs anerkennen. Anerkannte Ansprüche basieren auf bestimmten
sozialen Beziehungen die Menschen miteinander verbinden (Ash, 1964, S.
46). Allgemein zugestandene Ansprüche hängen von sozialen Beziehungen
ab, welche von den materiellen Bedingungen des Lebens einer Gesellschaft
abhängen. Deshalb sind \enquote{Rechte}, als das, was Menschen
berechtigt sind voneinander zu erwarten, letztlich auf der ökonomischen
Basis einer Gesellschaft begründet (1964, S. 47).

Im Kapitalismus als einer warenproduzierenden Gesellschaft, werden die
sozialen Beziehungen zwischen Menschen (von denen Ansprüche anhängen)
\enquote{entmenschlicht} und in Beziehungen zwischen Dingen verwandelt.
So ergibt sich die Tendenz, dass Ansprüche gegenüber anderen Menschen
die Form gerichtlich beglaubigter Rechte auf Dinge annehmen. Die soziale
Tatsache des Mehrwerts erscheint deshalb nicht länger als Ausbeutung des
einen Teils der Gesellschaft durch den anderen, sondern ist als
rationale Konsequenz natürlicher Eigentumsgesetze gerechtfertigt (Ash,
1964, S. 53). So ist die Idee, dass alle Menschen gleich sind selber ein
Korrelat der Gleichsetzung menschlicher Arbeit im Allgemeinen mit der
Quelle und dem Maßstab des Werts. Umgekehrt heißt das, dass der Begriff
menschlicher Gleichheit die Beständigkeit des öffentlichen Vorurteils
erlangen musste, bevor die wirkliche Natur des Werts angemessen
anerkannt werden konnte (1964, S. 51).

Eine Analyse der materiellen Grundlage von Rechten liefert uns nicht nur
ein Verständnis bestehender Rechtsverhältnisse, sondern ermöglicht es
uns auch, Konzeptionen des menschlichen Zusammenlebens einzuschätzen.
Denn die materialistischen Analyse von Rechtsverhältnissen verweist auf
den materiellen Charakter von Veränderung. Der folgende Abschnitt geht
näher auf diese Konsequenz ein und zeigt, dass Held aufgrund seines
idealistischen Verständnisses von Rechten auch Veränderung nur
idealistisch denken kann und umgekehrt.

\subsection{Der materielle Charakter von
Veränderung}\label{der-materielle-charakter-von-veruxe4nderung}

Held will \enquote{die menschliche Handlungsfähigkeit nicht als
bloße{[}n{]} Ausdruck einer bestimmten Teleologie, des Schicksals oder
einer Tradition verstanden} wissen. Die menschliche Handlungsfähigkeit
\enquote{muss eher als das Vermögen gedacht werden, sich für eine von
diesen Mustern abweichende Handlung entscheiden zu können -- also die
Fähigkeit, die menschliche Gemeinschaft nicht nur zu akzeptieren,
sondern sie auch, im Zusammenhang mit den Entscheidungen anderer, durch
das eigene Handeln zu formen} (Held, 2013, S. 67). In dieser Redeweise
wird deutlich, dass Helds Verständnis der Funktionsweise
gesellschaftlicher Prozesse kopfsteht, weil er sie vom Kopf her denkt.
Mit \enquote{Teleologie}, \enquote{Schicksal} und \enquote{Tradition}
spricht Held \emph{ideelle} Abhängigkeiten des menschlichen Handelns an
und lässt die \emph{wirklichen} Abhängigkeiten unbeachtet. Das zeigt,
das Held die Schranken des menschlichen Handelns eher in Ideen als in
ihrer \emph{wirklichen} Grundlage sieht.

Aber diese \emph{ideellen} Abhängigkeiten sind ihrerseits nur ein
Ausdruck der \emph{wirklichen}, die sich notwendig aus der materiellen
Produktion des Lebens der Menschen ergeben. Im Zuge dieser Produktion
gehen die Menschen \enquote{bestimmte, notwendige, von ihrem Willen
unabhängige Verhältnisse} ein (Marx, 1859/1961, S. 8). Das materielle
Leben der Menschen ist gesellschaftlich, weil es gesellschaftlich
produziert wird und bringt Verhältnisse mit sich, die von unserem Willen
unabhängig sind. Die Verhältnisse in denen wir leben sind
kapitalistische Verhältnisse. Die kapitalistische Produktionsweise
zeichnet sich durch eine global-\emph{gesellschaftliche} Produktion, bei
gleichzeitiger \emph{privater} Aneignung ihres Mehrprodukts aus. Das
Privateigentum an Produktionsmitteln macht ihre Besitzer zu Käufern der
Ware Arbeitskraft und einen übergroßen Teil der Gesellschaft zu
Verkäufern derselben. Ein System, indem zwar gesellschaftliche
produziert aber privat angeeignet wird, produziert nicht zum Wohle der
Gesellschaft, sondern im Sinne der privaten Aneignung.

Die Rechte und Pflichten die Held formuliert, müssen sich aus den
wirklichen materiellen Verhältnissen ergeben und werden nicht dadurch
verwirklicht, dass der Kosmopolitanismus sie \enquote{spezifiziert} oder
\enquote{aufzeigt} (Held, 2013, S. 74f.). Insofern sie unter den
gegebenen Bedingungen verwirklicht werden, stellen sie keinen
Widerspruch zu diesen Bedingungen dar. Insofern sie diesen
widersprechen, können sie nicht verwirklicht werden. Helds Prinzipien
stehen zwar in entscheidenden Punkten mit den herrschenden Verhältnissen
im Widerspruch, aber sie sind weder dazu geeignet etwas an diesen
Verhältnissen zu ändern, noch diese auch nur herauszufordern. Der
materiellen Grundlage von Veränderungsprozessen ist auch nicht damit
Rechnung getragen, wenn Held von \enquote{strukturelle Ungleichheiten}
spricht, \enquote{die manche Menschen daran hindern {[}\ldots{}{]} ihre
vitalsten Bedürfnisse zu befriedigen} oder, dass die Anerkennung des
\enquote{moralisch gleichen Wert einer jeden Person ein politisches,
herrschaftsfreies Verfahren notwendig macht} (Held, 2013, S. 68). Denn
bei Held sind die Ursachen \enquote{struktureller Ungleichheiten} und
die Bedingung eines \enquote{herrschaftsfreien Verfahrens} nicht
Gegenstand einer materialistischen Untersuchung sondern der politischen
Deliberation.

Dazu passt, dass Held sich weigert vom Kapitalismus zu sprechen. Das
Wort Kapitalismus kommt im ganzen Buch nur einmal vor, und auch das nur
eher am Rande in der vagen Redeweise eines \enquote{angloamerikanischen
Kapitalismus-Modells} und seiner \enquote{strukturellen Schwäche} (2013,
S. 17). Betrachten wir die Konsequenz dieser Verweigerung für sein
siebtes Prinzip der Vermeidung schweren Leids. Dieses
\enquote{Leitprinzip der sozialen Gerechtigkeit}, räumt \enquote{solange
den dringlichsten Bedürfnissen Vorrang} ein, \enquote{solange nicht die
ersten sechs Prinzipien de facto und de jure für alle Menschen gegeben
sind.} Aber \enquote{den dringlichsten Bedürfnissen Vorrang} einzuräumen
zielt, wenn richtig verstanden, ins Herz des Kapitalismus. Und solange
dieses Postulat nicht als Forderung gemeint ist, und dafür gibt es bei
Held keine Anzeichen, \enquote{alle Verhältnisse umzuwerfen, in denen
der Mensch ein erniedrigtes, ein geknechtetes, ein verlassenes, ein
verächtliches Wesen ist} (Marx, 1844/1981b, S. 385), solange kann sein
siebtes Prinzip nur als Lippenbekenntnis aufgefasst werden.

\section{Fazit}\label{fazit}

Held sieht die Lösung der globalen Probleme in seinem
Kosmopolitanismusmodell. Aber er liefert keine Analyse der Ursachen
dieser Probleme. Stellt Helds Vorstellung des Kosmopolitanismus eine
Lösung für die von ihm konstatierten Probleme dar? Nehmen wir sein
erstes Prinzip, wonach der Kosmopolitanismus \enquote{jede Person als
autonomen moralischen Akteur mit einem rechtmäßigen Anspruch auf gleich
Würde und Beachtung} anerkennt. Inwiefern verhindert dieses Prinzip,
dass höhere oder niedere Würde, mehr oder weniger Beachtung für eine
Person aus ihrer lokalen Zugehörigkeit gefolgert werden kann? Werden
viele Kriege nicht gerade im Namen der Menschenrechte geführt? Ging es
beim Bundeswehreinsatz in Afghanistan nicht vor allem um die Rechte und
das Leben der afghanischen Frauen? Wollte George W. Bush nicht
Demokratie im Irak stationieren, respektive abwerfen? Wenn dem so war,
dann ist die Hölle, die mit der Besetzung Afghanistans und der des Iraks
für die dortigen Menschen hereinbrach, zumindest kein Problem davon,
dass hier irgendjemand als minderwertig betrachtet wurde. Wenn dem nicht
so war, dann ging es vielleicht gar nicht um Menschenrechte? Dann ging
es vielleicht um was ganz anderes, um Interessen, die sich nicht von
einem solchen Prinzip aufhalten lassen.

\newpage

\section*{Eigenständigkeitserklärung}\label{eigenstuxe4ndigkeitserkluxe4rung}
\addcontentsline{toc}{section}{Eigenständigkeitserklärung}

Hiermit erkläre ich, dass ich die vorliegende Arbeit \emph{Die
materielle Grundlage von Rechten und gesellschaftlicher Veränderung.
Eine Kritik an David Helds Vorstellung des Kosmopolitanismus}
selbständig verfasst und keine anderen als die angegebenen Hilfsmittel
benutzt habe. Sie wurde im SoSe 2016 als Prüfungsleitung in der
Veranstaltung \enquote{Kosmopolitanismus und Modelle globaler Ordnung}
von Dr.~Rosa Sierra erstellt. Die Stellen der Arbeit, die anderen
Quellen im Wortlaut oder auch nur dem Sinn nach entnommen wurden, sind
durch Angaben der Herkunft kenntlich gemacht. Die vorliegende Arbeit
wurde bisher in keinem anderen Kontext als Prüfungsleistung vorgelegt.
\vspace{3em}

\noindent Frankfurt / M., 23. August 2016

\newpage

\section*{Bibliographie}\label{bibliographie}
\addcontentsline{toc}{section}{Bibliographie}

\vspace{-1em}

\indent
\vspace{-2em} \setlength{\parindent}{-0.2in}
\setlength{\leftskip}{0.2in} \setlength{\parskip}{8pt} \singlespacing

\hypertarget{refs}{}
\hypertarget{ref-ash1964}{}
Ash, W. (1964). \emph{Marxism and Moral Concepts}. New York: Monthly
Review Press.

\hypertarget{ref-held2013}{}
Held, D. (2013). \emph{Kosmopolitanismus: Ideal und Wirklichkeit}. (E.
Weiler, Übers.). Freiburg: Karl Alber.

\hypertarget{ref-lohmann1999}{}
Lohmann, G. (1999). Karl Marx fatale Kritik der Menschenrechte. In K. G.
Ballestrem, V. Gerhardt, H. Ottmann, \& M. P. Thompson (Hrsg.),
\emph{Politisches Denken. Jahrbuch} (S. 91--104). Stuttgart; Weimar: J.
B. Metzler.

\hypertarget{ref-marx1961}{}
Marx, K. (1961). Zur Kritik der Politischen Ökonomie. In Institut für
Marxismus-Leninismus beim ZK der SED (Hrsg.), \emph{Marx-Engels-Werke
Bd. 13} (1. Aufl., S. 7--11). Berlin: Dietz. (Original erschienen:
1859)

\hypertarget{ref-marx1981}{}
Marx, K. (1981a). Zur Judenfrage. In Institut für Marxismus-Leninismus
beim ZK der SED (Hrsg.), \emph{Marx-Engels-Werke Bd. 1} (13 überarb., S.
347--377). Berlin: Dietz. (Original erschienen: 1844)

\hypertarget{ref-marx1981b}{}
Marx, K. (1981b). Zur Kritik der Hegelschen Rechtsphilosophie.
Einleitung. In Institut für Marxismus-Leninismus beim ZK der SED
(Hrsg.), \emph{Marx-Engels-Werke Bd. 1} (13 überarb., S. 378--391).
Berlin: Dietz. (Original erschienen: 1844)

\hypertarget{ref-ullrich1974}{}
Ullrich, H. (1974). Karl Marx' \enquote{Zur Judenfrage}. \emph{Deutsche
Zeitschrift für Philosophie}, \emph{22}(8), 984--998.

\end{document}
